% Exemples

%%% Exemple pour la définition du Process Hitting %%%
\def \exphdef {
\path[use as bounding box] (-0.5,-0.5) rectangle (6.5,4.5);

\TSort{(0,3)}{a}{2}{l}
\TSort{(0,0)}{b}{2}{l}
\TSort{(6,1)}{z}{3}{r}

\THit{a_1}{}{z_1}{.west}{z_2}
\THit{b_1}{}{z_0}{.west}{z_1}
\THit{a_0}{out=250,in=200,selfhit}{a_0}{.west}{a_1}

\path[bounce,bend left]
\TBounce{z_0}{}{z_1}{.south}
\TBounce{z_1}{}{z_2}{.south}
\TBounce{a_0}{}{a_1}{.south}
;
}



%%% Exemple pour la coopération %%%
\def \exphcoop {
\path[use as bounding box] (-0.5,-0.5) rectangle (6.5,4.5);

% Actions de màj grisées
\only<6->{
\THit{a_1}{ulhit,color=lightgray}{ab_0}{.west}{ab_2}
\THit{a_1}{ulhit,color=lightgray}{ab_1}{.west}{ab_3}
\path[bounce,bend left,pulhit] \TBounce{ab_0}{bulhit}{ab_2}{.south} \TBounce{ab_1}{bulhit}{ab_3}{.south} ;
}

\only<7->{
\THit{a_0}{ulhit}{ab_2}{.west}{ab_0}
\THit{a_0}{ulhit}{ab_3}{.west}{ab_1}
\path[bounce,bend right,pulhit] \TBounce{ab_2}{bulhit}{ab_0}{.north} \TBounce{ab_3}{bulhit}{ab_1}{.north} ;
}

\only<8->{
\THit{b_0}{ulhit}{ab_3}{.west}{ab_2}
\THit{b_0}{ulhit}{ab_1}{.west}{ab_0}
\THit{b_1}{ulhit}{ab_0}{.west}{ab_1}
\THit{b_1}{ulhit}{ab_2}{.west}{ab_3}
\path[bounce,bend right,pulhit] \TBounce{ab_1}{bulhit}{ab_0}{.north} \TBounce{ab_3}{bulhit}{ab_2}{.north} ;
\path[bounce,bend left,pulhit] \TBounce{ab_0}{bulhit}{ab_1}{.south} \TBounce{ab_2}{bulhit}{ab_3}{.south} ;
}

% Sortes
\TSort{(0,3)}{a}{2}{l}
\TSort{(0,0)}{b}{2}{l}
\TSort{(6,1)}{z}{3}{r}

% Deux actions disjointes en exemple
\only<2-3>{
\THit{a_1}{}{z_1}{.north west}{z_2}
\path[bounce,bend left]
\TBounce{z_1}{}{z_2}{.south} ;

\THit{b_0}{}{z_1}{.west}{z_2}
\path[bounce,bend left=55]
\TBounce{z_1}{}{z_2}{.south west} ;
}

% Processus d'exemple
\TState{3}{a_1,b_1,z_1}

% Sorte coopérative et arcs
\only<4->{
\TSetTick{ab}{0}{00}
\TSetTick{ab}{1}{01}
\TSetTick{ab}{2}{10}
\TSetTick{ab}{3}{11}
\TSort{(3,0.5)}{ab}{4}{l}
}

% Arcs de màj noirs de la sc
\only<5>{
\THit{a_1}{thick}{ab_0}{.west}{ab_2}
\THit{a_1}{thick}{ab_1}{.west}{ab_3}
\path[bounce,thick,bend left] \TBounce{ab_0}{thick}{ab_2}{.south} \TBounce{ab_1}{thick}{ab_3}{.south} ;
}

\only<6>{
\THit{a_0}{thick}{ab_2}{.west}{ab_0}
\THit{a_0}{thick}{ab_3}{.west}{ab_1}
\path[bounce,thick,bend right] \TBounce{ab_2}{thick}{ab_0}{.north} \TBounce{ab_3}{thick}{ab_1}{.north} ;
}

\only<7>{
\THit{b_0}{thick}{ab_3}{.west}{ab_2}
\THit{b_0}{thick}{ab_1}{.west}{ab_0}
\THit{b_1}{thick}{ab_0}{.west}{ab_1}
\THit{b_1}{thick}{ab_2}{.west}{ab_3}
\path[bounce,thick,bend right] \TBounce{ab_1}{thick}{ab_0}{.north} \TBounce{ab_3}{thick}{ab_2}{.north} ;
\path[bounce,thick,bend left] \TBounce{ab_0}{thick}{ab_1}{.south} \TBounce{ab_2}{thick}{ab_3}{.south} ;
}

% État d'exemple pour màj de la sc
\TState{8-9}{a_1,b_0}
\TState{10}{a_1,b_0,ab_0,ab_1,ab_2,ab_3}
\TState{11}{a_1,b_0,ab_2}
\only<9-11>{
\THit{a_1}{}{ab_0}{.west}{ab_2}
\THit{a_1}{}{ab_1}{.west}{ab_3}
\THit{b_0}{}{ab_3}{.west}{ab_2}
\THit{b_0}{}{ab_1}{.west}{ab_0}
\path[bounce,bend left] \TBounce{ab_0}{}{ab_2}{.south} \TBounce{ab_1}{}{ab_3}{.south} ;
\path[bounce,bend right] \TBounce{ab_1}{}{ab_0}{.north} \TBounce{ab_3}{}{ab_2}{.north} ;
}

% État d'exemple pour action de la sc
\TState{12}{a_1,b_0,z_1,ab_2}
\TState{13-14}{a_1,b_0,z_2,ab_2}

% Arc sortant de la sc
\only<12-14>{
\THit{ab_2}{thick}{z_1}{.west}{z_2}
\path[bounce,bend left,thick] \TBounce{z_1}{thick}{z_2}{.south} ;
}

% Arc sortant de la sc
%\only<15->{
%\THit{ab_2}{}{z_1}{.west}{z_2}
%\path[bounce,bend left] \TBounce{z_1}{}{z_2}{.south} ;
%}

}



%%% Exemple pour l'inférence %%%
\def \exphinf {
% Sortes
\TSort{(0,3)}{a}{2}{l}
\TSort{(0,0)}{b}{2}{l}
\TSort{(6,0)}{z}{3}{r}

% Sorte coopérative et arcs
\TSetTick{ab}{0}{00}
\TSetTick{ab}{1}{01}
\TSetTick{ab}{2}{10}
\TSetTick{ab}{3}{11}
\TSort{(3,0)}{ab}{4}{l}

% Actions de màj grisées
\THit{a_1}{ulhit}{ab_0}{.west}{ab_2}
\THit{a_1}{ulhit}{ab_1}{.west}{ab_3}
\path[bounce,bend left,pulhit] \TBounce{ab_0}{bulhit}{ab_2}{.south} \TBounce{ab_1}{bulhit}{ab_3}{.south};

\THit{a_0}{ulhit}{ab_2}{.west}{ab_0}
\THit{a_0}{ulhit}{ab_3}{.west}{ab_1}
\path[bounce,bend right,pulhit] \TBounce{ab_2}{bulhit}{ab_0}{.north} \TBounce{ab_3}{bulhit}{ab_1}{.north};

\THit{b_0}{ulhit}{ab_3}{.west}{ab_2}
\THit{b_0}{ulhit}{ab_1}{.west}{ab_0}
\THit{b_1}{ulhit}{ab_0}{.west}{ab_1}
\THit{b_1}{ulhit}{ab_2}{.west}{ab_3}
\path[bounce,bend right,pulhit] \TBounce{ab_1}{bulhit}{ab_0}{.north} \TBounce{ab_3}{bulhit}{ab_2}{.north};
\path[bounce,bend left,pulhit] \TBounce{ab_0}{bulhit}{ab_1}{.south} \TBounce{ab_2}{bulhit}{ab_3}{.south};

% Arcs sortant de la sc
\THit{ab_2}{ulhit}{z_1}{.north west}{z_2}
\THit{ab_2}{ulhit}{z_0}{.west}{z_1}
\path[bounce,bend left,pulhit] \TBounce{z_1}{bulhit}{z_2}{.south} \TBounce{z_0}{bulhit}{z_1}{.south};

\THit{ab_3}{ulhit}{z_2}{.west}{z_1}
\THit{ab_3}{ulhit}{z_0}{.west}{z_1}
\THit{ab_1}{ulhit}{z_2}{.west}{z_1}
\THit{ab_1}{ulhit}{z_0}{.west}{z_1}
\path[bounce,bend left,pulhit] \TBounce{z_2}{bulhit,bend right}{z_1}{.north};

\THit{ab_0}{ulhit}{z_2}{.west}{z_1}
\THit{ab_0}{ulhit}{z_1}{.south west}{z_0}
\path[bounce,bend right,pulhit] \TBounce{z_2}{bulhit}{z_1}{.north} \TBounce{z_1}{bulhit}{z_0}{.north};

}



%%% Exemple pour l'inférence (sans arcs grisés) %%%
\def \exphinfblack {
% Sortes
\TSort{(0,3)}{a}{2}{l}
\TSort{(0,0)}{b}{2}{l}
\TSort{(6,0)}{z}{3}{r}

% Sorte coopérative et arcs
\TSetTick{ab}{0}{00}
\TSetTick{ab}{1}{01}
\TSetTick{ab}{2}{10}
\TSetTick{ab}{3}{11}
\TSort{(3,0)}{ab}{4}{l}

% Actions de màj grisées
\THit{a_1}{}{ab_0}{.west}{ab_2}
\THit{a_1}{}{ab_1}{.west}{ab_3}
\path[bounce,bend left] \TBounce{ab_0}{}{ab_2}{.south} \TBounce{ab_1}{}{ab_3}{.south};

\THit{a_0}{}{ab_2}{.west}{ab_0}
\THit{a_0}{}{ab_3}{.west}{ab_1}
\path[bounce,bend right] \TBounce{ab_2}{}{ab_0}{.north} \TBounce{ab_3}{}{ab_1}{.north};

\THit{b_0}{}{ab_3}{.west}{ab_2}
\THit{b_0}{}{ab_1}{.west}{ab_0}
\THit{b_1}{}{ab_0}{.west}{ab_1}
\THit{b_1}{}{ab_2}{.west}{ab_3}
\path[bounce,bend right] \TBounce{ab_1}{}{ab_0}{.north} \TBounce{ab_3}{}{ab_2}{.north};
\path[bounce,bend left] \TBounce{ab_0}{}{ab_1}{.south} \TBounce{ab_2}{}{ab_3}{.south};

% Arcs sortant de la sc
\THit{ab_2}{}{z_1}{.north west}{z_2}
\THit{ab_2}{}{z_0}{.west}{z_1}
\path[bounce,bend left] \TBounce{z_1}{}{z_2}{.south} \TBounce{z_0}{}{z_1}{.south};

\THit{ab_3}{}{z_2}{.west}{z_1}
\THit{ab_3}{}{z_0}{.west}{z_1}
\THit{ab_1}{}{z_2}{.west}{z_1}
\THit{ab_1}{}{z_0}{.west}{z_1}
\path[bounce,bend left] \TBounce{z_2}{,bend right}{z_1}{.north};

\THit{ab_0}{}{z_2}{.west}{z_1}
\THit{ab_0}{}{z_1}{.south west}{z_0}
\path[bounce,bend right] \TBounce{z_2}{}{z_1}{.north} \TBounce{z_1}{}{z_0}{.north};

}



%%% Exemple 2 pour l'inférence (projections) %%%
\def \exphinfproj {
% Sortes
\TSort{(0,3)}{a}{2}{l}
\TSort{(0,0)}{b}{2}{l}
\TSort{(6,1)}{z}{2}{r}

% Sorte coopérative et arcs
\TSetTick{ab}{0}{00}
\TSetTick{ab}{1}{01}
\TSetTick{ab}{2}{10}
\TSetTick{ab}{3}{11}
\TSort{(3,0)}{ab}{4}{l}

% Actions de màj grisées
\THit{a_1}{ulhit}{ab_0}{.west}{ab_2}
\THit{a_1}{ulhit}{ab_1}{.west}{ab_3}
\path[bounce,bend left,pulhit] \TBounce{ab_0}{bulhit}{ab_2}{.south} \TBounce{ab_1}{bulhit}{ab_3}{.south} ;

\THit{a_0}{ulhit}{ab_2}{.west}{ab_0}
\THit{a_0}{ulhit}{ab_3}{.west}{ab_1}
\path[bounce,bend right,pulhit] \TBounce{ab_2}{bulhit}{ab_0}{.north} \TBounce{ab_3}{bulhit}{ab_1}{.north} ;

\THit{b_0}{ulhit}{ab_3}{.west}{ab_2}
\THit{b_0}{ulhit}{ab_1}{.west}{ab_0}
\THit{b_1}{ulhit}{ab_0}{.west}{ab_1}
\THit{b_1}{ulhit}{ab_2}{.west}{ab_3}
\path[bounce,bend right,pulhit] \TBounce{ab_1}{bulhit}{ab_0}{.north} \TBounce{ab_3}{bulhit}{ab_2}{.north} ;
\path[bounce,bend left,pulhit] \TBounce{ab_0}{bulhit}{ab_1}{.south} \TBounce{ab_2}{bulhit}{ab_3}{.south} ;

% Arcs sortant de la sc
\THit{ab_3}{ulhit}{z_0}{.west}{z_1}
\path[bounce,bend left,pulhit] \TBounce{z_0}{bulhit}{z_1}{.south} ;

\THit{ab_0}{ulhit}{z_1}{.west}{z_0}
\path[bounce,bend right,pulhit]\TBounce{z_1}{bulhit}{z_0}{.north} ;
}



%%% Exemple sans sorte coopérative pour l'inférence %%%
\def \exphinfprojssc {
% Sortes
\TSort{(0,3)}{a}{2}{l}
\TSort{(0,0)}{b}{2}{l}
\TSort{(6,0)}{z}{3}{r}

\THit{a_1}{ulhit}{z_0}{.west}{z_1}
\THit{a_1}{ulhit}{z_1}{.north west}{z_2}
\THit{a_0}{ulhit}{z_1}{.south west}{z_0}
\THit{a_0}{ulhit}{z_2}{.west}{z_1}
\path[bounce,bend left,pulhit] \TBounce{z_0}{bulhit}{z_1}{.south} \TBounce{z_1}{bulhit}{z_2}{.south}
  \TBounce{z_1}{bulhit,bend right}{z_0}{.north} \TBounce{z_2}{bulhit,bend right}{z_1}{.north} ;

\THit{b_0}{ulhit}{z_0}{.west}{z_1}
\THit{b_0}{ulhit}{z_1}{.north west}{z_2}
\THit{b_1}{ulhit}{z_1}{.south west}{z_0}
\THit{b_1}{ulhit}{z_2}{.west}{z_1}
%\path[bounce,bend left,pulhit] \TBounce{z_0}{bulhit}{z_1}{.south} \TBounce{z_1}{bulhit}{z_2}{.south}
%  \TBounce{z_1}{bulhit,bend right}{z_0}{.north} \TBounce{z_2}{bulhit,bend right}{z_1}{.north} ;
}



%%% Exemple atteignabilité
\def \exatt {
\path[use as bounding box] (-1,-3) rectangle (7,2.7);
\TSort{(0,0)}{a}{2}{l}
\TSort{(3,0)}{b}{3}{l}
\TSort{(6,0)}{d}{3}{r}
\TSort{(2,-2)}{c}{2}{b}

\THit{a_0}{}{c_0}{.north}{c_1}
\THit{a_1}{}{b_1}{.west}{b_0}
\THit{c_1}{bend left=20pt}{b_0}{.west}{b_1}
\THit{b_1.south west}{->}{a_0}{.east}{a_1}
\THit{b_0}{}{d_0}{.west}{d_1}
\THit{b_1}{}{d_1}{.west}{d_2}
\THit{d_1}{}{b_0}{.north east}{b_2}
\THit{c_1}{bend right=80pt,distance=80pt}{d_1}{.east}{d_0}
\THit{b_2}{distance=120pt,out=30,in=40}{d_0}{.east}{d_2}

\path[bounce,bend left]
\TBounce{d_0}{}{d_1}{.south}
\TBounce{d_1}{}{d_2}{.south}
\TBounce{c_0}{}{c_1}{.west}
\TBounce{b_0}{}{b_1}{.south}
\TBounce{d_1}{}{d_0}{.north}
;
\path[bounce,bend right]
\TBounce{a_0}{}{a_1}{.south}
\TBounce{b_0}{}{b_2}{.south}
\TBounce{b_1}{}{b_0}{.north}
\TBounce{d_0}{bend right=50pt,distance=40pt}{d_2}{.south}
;
}



%%% Figure de présentation de l'analyse d'atteignabilité
\def \figsa {
\begin{tikzpicture}
\path[use as bounding box] (-5,-3.5) rectangle (5,3.5);
\definecolor{r2}{RGB}{238,10,38}

\path<2->[shading=1, inner color=r2, outer color=white] (3.5,-2.8) -- (4.4,3.2) -- (0,3) -- (-4.5,1.4) -- (-2.5,-2.5) -- (0,-3.6) -- (2.8,-2.8);
%\path<2->[shading, inner color=r2, outer color=white, border color=white] (2.8,-2.8) -- (4.5,4.5) -- (0,3.9) -- (-4.5,1.8) -- (-5,-3) -- (0,-3.2) -- (2.8,-2.8);
\draw<2->[thick,fill=white] (2.5,-2.1) -- (3,2.5) -- (-2.7,1.3) -- (-2,-2) -- (2.5,-2.1);
\draw<6->[thick,fill=lightyellow] (2.5,-2.1) -- (3,2.5) -- (-2.7,1.3) -- (-2,-2) -- (2.5,-2.1);

\node<2->[text width=3.5cm, color=red] (s1) at (-5,2) {Over-Approximation};
\path<2->[->,very thick,color=red] (s1.south) edge (-3.5,1.2);
%\node<2->[text width=3cm,color=black] (i1) at (3.7,.2) {$\Rightarrow$};
\node<2->[text width=3cm,color=black] (q) at (4.5,.2) {$\neg Q$};

\draw<4->[thick, fill=green] (.5,-.8) -- (1,0) -- (.3,1) -- (-1,.5) -- (-.5,-.5) -- (.5,-.8);
\node<4->[text width=3.5cm,color=darkgreen] (s2) at (5.2,-1.5) {Under-Approximation};
\node<4->[text width=3cm,color=black] (p) at (1.8,.2) {$P$};
%\node<4->[text width=3cm,color=black] (i1) at (2.25,.2) {$\Rightarrow$};

% reaching set
\node[text width=3cm,color=darkcyan] (s) at (1.8,1.7) {Exact solution};
\node<1->[text width=3cm,color=darkcyan] (s0) at (0,0) {};
\draw[color=darkcyan, thick] (0,0) ellipse (2 and 1.5);
%\path<1>[draw=white] (2.8,-2.8) -- (4.5,4.5) -- (0,3.9) -- (-4.5,1.8) -- (-5,-3) -- (-2.5,-3.5) -- (0,-3.2) -- (2.8,-2.8);
\node[text width=3cm,color=black] (r) at (2.8,.2) {$R$};

\path<4->[->,very thick,color=darkgreen] (s2) edge (.6,-.4);

\tikzstyle{point}=[circle,draw=blue,fill=blue,minimum size=5pt,inner sep=0pt]

%\only<5->{
\only<3->{
\node[point] at (-2.4,-2) {};
\node[point] at (-2,2) {};
}
\only<5->{
\node[point] at (0,0) {};
}
\only<7->{
\node[point] at (-.5,-1.1) {};
\node[point] at (2.5,1) {};
}
%}

\end{tikzpicture}
}
